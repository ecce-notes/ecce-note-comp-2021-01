
\usepackage{amssymb}
%% \usepackage{amsthm}

\usepackage{caption}
\usepackage{xspace}
\usepackage{longtable}
\usepackage{pdflscape}
\usepackage{multirow}
\usepackage{ragged2e}
\usepackage{booktabs}
\usepackage{hyperref}
\usepackage{tikz}
\usetikzlibrary{arrows,chains,shapes,matrix,scopes,positioning,fadings,backgrounds,fit,mindmap,trees,decorations.markings,decorations.pathreplacing,calc}
\usetikzlibrary{decorations.pathmorphing,decorations.markings,trees,arrows.meta}

\newcommand{\eA}{\mbox{$e-{\rm A}$}}
\newcommand{\jpsi}{\mbox{$J/\psi$}}
\newcommand{\mrwell}{\mbox{$\mu$RWELL}\xspace}
\newcommand{\PbWOiv}{\mbox{PbWO$_4$}\xspace}
\newcommand{\geant}{\mbox{\sc Geant4}\xspace}

%% The lineno packages adds line numbers. Start line numbering with
%% \begin{linenumbers}, end it with \end{linenumbers}. Or switch it on
%% for the whole article with \linenumbers.
\usepackage{lineno}
