%\textbf{\emph{Cristiano/Will}}
%
%\emph{This is intended to be a fairly brief section.
%AI/ML is expected to be integrated into all aspects of the Computing Plan so should appear throughout the document.
% }
%
%
%EIC 
%%has the possibility to incorporate AI from the start and 
%could be one of the first large-scale experiments where AI is systematically employed from the detector and design 

An example of such a new paradigm is the use of Artificial Intelligence (AI) and machine learning (ML) in its program. The EIC has the unique opportunity to be one of the first large-scale experiments to systematically incorporate and employ AI, starting from the detector design and R\&D phases. 
The relevance of AI for the EIC has been highlighted among the \textit{Opportunities for Computing} of the EIC Yellow Report \cite{eic_yellow_report_v1_1}. AI potentially permeates all aspects of the Computing Plan, and appears throughout this document; in fact, it already plays a significant role in the design and R\&D phases of the EIC~\cite{cisbani2020ai}.

%The first workshop on Artificial Intelligence for the Electron Ion Collider (AI4EIC) held at CFNS in September 2021 \cite{AI4EIC_workshop} has been a fundamental venue for the EIC community to start addressing how AI might contribute to advance research, design and operation of the future EIC across all the phases of the EIC schedule (see \cite{AI4EIC_future}).
%---- structure of AI4EIC 
%The workshop was structured into different sessions focusing on experimental applications of AI for \textit{Accelerator and Detector Design}, \textit{Simulations}, \textit{Reconstruction and Analysis}, \textit{Accelerator and Detector Control}, \textit{Detector Readout} and \textit{Computing Frontiers}. 



It is in fact clear the advantage of AI techniques in searching the optimal solutions in a complex multi-dimensional parameter space such as in the design of a detector. Detector design requires accurate simulations based on Geant4 \cite{ALLISON2016186}. Those simulations are typically compute intensive and another advantage of using AI is minimizing the computing budget needed to come up with an optimal solution (see, \textit{e.g.}, the results of an AI-supported design of a complex detector system like the dual-RICH \cite{cisbani2020ai}).
%AI can be also utilized for the optimization of materials used within detectors for improved performance. 

%Simulations - fast simulations - ATLAS example 

%The AI4EIC session on Simulations highlighted the need for faster simulations. The ATLAS Collaboration at CERN for example is already including a GAN \cite{creswell2018generative} in AtlFast3, the next generation of high-accuracy fast simulation in ATLAS \cite{atlas2021atlfast3}. 
%for fast simulations at intermediate energies.   
The EIC detector has unique features. For example, the backbone of particle identification is with imaging Cherenkov detectors \cite{he2020rich} (DIRC in the central region, mRICH in the e-going direction and dRICH in the h-going direction of the ECCE detector), which are typically characterized by compute intensive simulations and complex pattern recognition problems, that can benefit from novel AI-based approaches that have been recently proposed (see, \textit{e.g.}, \cite{fanelli2020deeprich}). %(see, \textit{e.g.}, \cite{ali2020gluex}).   
%
%
%
%Reconstruction - EIC detector has unique features. The backbone of PID is Cherenkov detectors. Compute intensive simulations and complex pattern recognition problems for imaging Cherenkov detectors like DIRC.  
 AI is also expected to play a role in high-level physics analysis, for example in jet physics to empower taggers for boosted jets and identify quark flavor within the jets (see ML4Jets workshops \cite{ml4jets}).    

%Another important aspect came from the Accelerator and Detector Control session of AI4EIC: in fact EIC can been envisaged as one the first experiments to be largely automated and endowed with intelligence in control workflows (see, \textit{e.g.}, \cite{Hydra2021} as discussed in Sec.~\ref{subsec:online_mon}).  

The EIC community is also moving towards a trigger-less Streaming Readout (SRO) approach to detector readout as discussed in Sec.~\ref{sec:online}. 
This represents a unique opportunity to streamline workflows for online and offline data processing in HEP and NP experiments, and to take advantage of emerging technologies such as \textit{e.g.}, heterogeneous computing with AI/ML as discussed in the Computing Frontiers session of AI4EIC.
Within the new paradigm of SRO, AI is expected to play a major role in providing fast alignment/calibration/reconstruction for near real-time analyses.

% What opportunities from “computing frontiers” in > 2030? 


%--- references to next sections go here ---

%--- few sentences to uncomment --- 

%--- Will: any addition?

%---- some numbers from:
%%CPU Resources
%%ECCE Tracker Optimization (Cristiano/Karthik)
%%DIRC Optimization Framework # of compute hours/configuration (Will/Andru)
%%Dual Rich (what used for the study in paper) (Cristiano)
%%General purpose simulation for generating training samples for projects listed below
%%???
%%GPU Resources
%%CLAS12 Tracking  (Gagik/ODU Students)
%%Hydra Online Monitoring Training (Kishan/Thomas)
%%Online track reconstruction?
%%???

%---- final summaries, keywords and citations 


