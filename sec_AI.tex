%\textbf{\emph{Cristiano/Will}}

%\emph{This is intended to be a fairly brief section.
%AI/ML is expected to be integrated into all aspects of the Computing Plan so should appear throughout the document.
% }


%EIC 
%%has the possibility to incorporate AI from the start and 
%could be one of the first large-scale experiments where AI is systematically employed from the detector and design 

The EIC can be one of the first large-scale experiments to systematically incorporate and employ Artificial Intelligence (AI) in its program, starting from the detector design and R\&D phases. 
The relevance of AI for the EIC has been highlighted among the \textit{Opportunities for Computing} of the EIC Yellow Report \cite{eic_yellow_report_v1_1}.

ECCE recognizes the important role that AI can play at various stages of the future EIC experiment and includes in its structure a working group dedicated to AI-based applications. 
As a result AI permeates all aspects of this Computing Plan and appears throughout the whole document.

The first workshop on Artificial Intelligence for the Electron Ion Collider (AI4EIC) held at CFNS in September 2021 \cite{AI4EIC_workshop} has been a fundamental venue for the EIC community to start addressing how AI might contribute to advance research, design and operation of the future EIC across all the phases of the EIC schedule (see \cite{AI4EIC_future}).
%---- structure of AI4EIC 
The workshop was structured into different sessions focusing on experimental applications of AI for \textit{Accelerator and Detector Design}, \textit{Simulations}, \textit{Reconstruction and Analysis}, \textit{Accelerator and Detector Control}, \textit{Detector Readout} and \textit{Computing Frontiers}. 
%
As a result of this discussion it turned out AI techniques are expected to play a role since the design and R\&D phases of EIC. 
It is in fact clear the advantage of AI techniques in searching the optimal solutions in a complex multi-dimensional parameter space such as in the design of a detector. Detector design requires accurate simulations based on Geant4 \cite{ALLISON2016186}. Those simulations are typically compute intensive and another advantage of using AI is minimizing the computing budget needed to come up with an optimal solution (see, \textit{e.g.}, the results of an AI-supported design of a complex detector system like the dual-RICH \cite{cisbani2020ai}).
%AI can be also utilized for the optimization of materials used within detectors for improved performance. 

%Simulations - fast simulations - ATLAS example 

%Reconstruction - EIC detector has unique features. The backbone of PID is Cherenkov detectors. Compute intensive simulations and complex pattern recognition problems for imaging Cherenkov detectors like DIRC.  

%AI is also expected to play a role in high-level physics analysis, for example in jet physics (cite ML4JETS) to empower taggers for boosted jets and identify quark flavor within the jets.  

% EIC can be one of the first ``automated''/autonomous experiments (see Control Session) control workflows 

% Streaming readout at EIC will further the convergence of online and offline analysis: AI will play a major role in providing fast alignment/calibration/reconstruction for near real-time analyses.

% What opportunities from “computing frontiers” in > 2030? 


--- references to next sections go here ---

--- few sentences to uncomment --- 

--- Will: any addition?

%---- some numbers from:
%%CPU Resources
%%ECCE Tracker Optimization (Cristiano/Karthik)
%%DIRC Optimization Framework # of compute hours/configuration (Will/Andru)
%%Dual Rich (what used for the study in paper) (Cristiano)
%%General purpose simulation for generating training samples for projects listed below
%%???
%%GPU Resources
%%CLAS12 Tracking  (Gagik/ODU Students)
%%Hydra Online Monitoring Training (Kishan/Thomas)
%%Online track reconstruction?
%%???

%---- final summaries, keywords and citations 


