In summary, the ECCE consortium has developed a comprehensive plan for EIC software and computing. We intend to deploy a federated computing model where multiple facilities are used, to take advantage of world wide computing resources. During the development of the ECCE proposal, a successful tiered model inspired by the WLCG was deployed. Computing sites at Brookhaven National Laboratory, Jefferson Laboratory, Oak Ridge National Laboratory, and Bates Laboratory processed large scale productions which were then stored centrally at two sites and distributed to smaller individual sites depending on user needs. These productions were run centrally with offline software that, where possible, utilized common developed software frameworks. Future iterations of the ECCE software stack will aim to incorporate more modern software development elements and packages that are actively worked on within the high energy and nuclear physics communities. Our online-offline computing model incorporates near real-time analysis goals where the final calibration and reconstruction of raw data occurs within 2-3 weeks of acquisition. Data will be buffered on disk at the tier 1 computing sites in accordance with the tiered model. We intend to integrate artificial intelligence and machine learning throughout this software and computing model, and in certain areas AI is already directing detector design and online data selection. Estimates for computing resource requirements have been provided; we emphasize that these estimates will become more accurate as time progresses and the EIC computing landscape becomes clearer. 