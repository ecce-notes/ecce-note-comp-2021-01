
\textbf{\emph{David L.}}

This document presents a proposed computing plan for the ECCE detector at the EIC\cite{eic_yellow_report_v1_1}. This includes estimations of the rates from the detector, the pipeline for processing and storing the data, and how the collaboration members will access the data. Software systems for monitoring, calibration, reconstruction, and analysis are discussed. Estimates of the computing and storage requirements are included.

At this point in time ECCE is still in the proposal stage. The development is being done by the ECCE proto-collaboration. Many details regarding the data pipeline and software systems have not been fully fleshed out. It is anticipated that once a formal ECCE collaboration is established that more rigorous designs will be developed. 

The EIC is anticipated to start producing data approximately a decade from now. As such, it will introduce new paradigms in large scale Nuclear Physics experiments. While we attempt to include some forward thinking plans in this regard, we necessarily do need to rely on past experience with other large experiments such as sPHENIX\cite{sphenix_computing_plan_2019} and LHCb\cite{CAMPORAPEREZ2016280} to serve as guides.