
%\textbf{\emph{?}}

%\emph{This needs to pull in numbers from the other sections and summarize the total needs in a couple of tables. These tables will be copied into the ECCE proposal.}

The EIC luminosity is projected to be between $10^{33}cm^{-2}s{-1}$ and $10^{34}cm^{-2}s{-1}$ (see sec. 2.10 of the Yellow Report\cite{eic_yellow_report_v1_1}). Assume 30 weeks of operation per year and 60\% accelerator operation efficiency once it is in full production mode. In the first years, however, we may expect fewer weeks of running and lower luminosity. Table \ref{tab:integrated_luminosity_by_year} lists a possible scenario used for the purposes of estimation in this section. We assume 100Gbps data rate to storage for $10^{34}cm^{-2}s{-1}$ and 60\% operational efficiency of the facility. All other rates are derived by scaling this value by the luminosity and efficiency values indicated in the table.


\begin{table}[htb!]
    \centering
    \begin{tabular}{c|c|c|c}
        \hline
        \hline
         \textbf{New Storage}       & year-1                & year-2                  & year-3                \\
        \hline
         Luminosity              & $10^{33}cm^{-2}s^{-1}$ & $2x10^{33}cm^{-2}s^{-1}$ & $10^{34}cm^{-2}s^{-1}$ \\
         \hline
         Weeks of Running        & 10                    & 20                      & 30                    \\
         \hline
         Operational efficiency    & 40\%                  & 50\%                    & 60\%                  \\
         \hline
         Data Rate to Storage    & 6.7Gbps               & 16.7Gbps                & 100Gbps               \\
         \hline
         Raw Data Storage (no duplicates) & 4PB          & 20PB                    & 181PB                 \\
         \hline
         Recon Storage          & 0.4PB                  & 2PB                    & 18PB                   \\
         \hline
         Total Storage (no duplicates) & 4.4PB           & 22PB                   & 200PB                  \\
         \hline
   \end{tabular}
    \caption{Estimate of raw data tape storage needed for first 3 years of EIC running (ECCE only). Values are estimates assuming ramp up to full luminosity  by year 3. Numbers for the first two years are estimated for the purposes of this exercise and do not come from an external source. n.b. each value represents \emph{only} the needs for data produced in that year and \emph{not} a cumulative total.}
    \label{tab:integrated_luminosity_by_year}
\end{table}

Temporary disk storage will be needed for raw data during the 3 week time span during which calibrations are derived and the raw data processed. In addition, disk storage will be needed for the reconstructed data that collaborators will be accessing for analysis. Table \ref{tab:disk_summary} gives estimates of the disk resources needed for the first 3 years of running. Note that the values in that table are cumulative and so represent the total amount of disk needed for each year which include recon data from previous years.


\begin{table}[htb!]
    \centering
    \begin{tabular}{c|c|c|c}
        \hline
        \textbf{Total Disk} & year-1 & year-2 & year-3 \\
        \hline
        \hline
        Disk (temporary)  &  1.2PB & 3.0PB & 18.1PB \\
        \hline
        Disk (permanent)    & 0.4PB & 2.4PB &	20.6PB \\
        \hline
        \textbf{TOTAL}          & 1.6PB &	5.4PB &	38.7PB \\
        \hline
    \end{tabular}
    \caption{Estimate of disk storage needed for first 3 years of EIC running (ECCE only). The temporary disk is used to hold raw data for a 3 week period while calibrations are derived and reconstruction is done. The permanent disk is for holding the reconstructed data. This will be cumulative so collaborators will have access to recon data from all years.}
    \label{tab:disk_summary}
\end{table}

The CPU required for processing the data is very difficult to estimate with any accuracy better than the order of magnitude. Nonetheless, an attempt is made here to provide such a ballpark estimate. Table \ref{tab:cpu_summary} provides summarizes the important values. The 5.4s/ev comes from estimating an average of 3 hours for reconstruction of 2k events of ECCE data. The numbers for ECCE CPU mainly come from the simulation campaigns run for proposal development which include combined simulation and reconstruction. The times to process 2k events ranged from 2 to 9 hours depending on the collision type and the CPU type that the job was processed on. This corresponds to a range of roughly 4s/ev to 16s/ev. The reconstruction only part is considered to be half of the roughly 6 hour average time to simulate 2k events. By way of comparison, sPHENIX estimates 15s/ev for Au+Au scattering and 10.4s/ev for p+p scattering (see section 5.2 of \cite{sphenix_computing_plan_2019}). Thus, 5.4s/ev is assumed to be at least the right order of magnitude. The event size of 250kB is also a rough average based on the ECCE DST files for several configurations simulated in the major proposal campaigns. It is used, along with the numbers for the Raw Data Storage from table \ref{tab:integrated_luminosity_by_year}, to calculate the number of events produced in each year. The CPU needed for calibration is estimated to be roughly 5\% of that needed for full reconstruction. It is noted that the sPHENIX Computing Plan estimates this to be 25\%. The final line in table \ref{tab:cpu_summary} estimates the number of CPU cores needed to process the data for each year assuming it can be done over a 30 week period. This would mean in year-3 there would be enough CPU to keep up with the raw data production rate. In earlier years, this would not be needed as the production times are much shorter.

\begin{table}[htb!]
    \centering
    \begin{tabular}{c|c|c|c}
        \hline
        CPU Compute & year-1 & year-2 & year-3 \\
        \hline
        \hline
        Recon process time/core	& 5.4s/ev	& 5.4s/ev	& 5.4s/ev \\
        \hline
        Event size	& 250kB	& 250 kB & 250kB \\
        \hline
        Number of events produced &	16B	& 81B & 726B \\
        \hline
        CPU-core hours (recon-only, 1 pass)	& 24Mcore-hrs	& 121Mcore-hrs &	1089Mcore-hrs \\
        \hline
        CPU-core hours (calib-only) &	1.2Mcore-hrs &	6.0Mcore-hrs &	54.4Mcore-hrs \\	
        \hline
        Cores needed to process in 30 weeks	& 5k &	25k &	227k \\
        \hline
    \end{tabular}
    \caption{Estimates of CPU needed for reconstruction of raw data. The number of seconds per event is highly dependent on the type of processor being used. Number of events comes from total raw data storage estimate in table \ref{tab:integrated_luminosity_by_year}. Calibration is assumed to be 5\% of reconstruction time.}
    \label{tab:cpu_summary}
\end{table}

% \begin{table}[htb!]
%     \centering
%     \begin{tabular}{c|c|c|c}
%         \hline
%         GPU Compute(Gcore-hr) & year-1 & year-2 & year-3 \\
%         \hline
%         \hline
%         Online   & & & \\
%         \hline
%         Offline Recon. & & & \\
%         \hline
%         Analysis  & & & \\
%         \hline
%         AI/ML    & & & \\
%         \hline
%         \textbf{TOTAL} & & & \\
%         \hline
%     \end{tabular}
%     \caption{Caption}
%     \label{tab:gpu_summary}
% \end{table}


