
The ECCE consortium plans to deploy a federated computing model for the EIC where multiple facilities are used. ECCE recognizes the need for a global EIC model and intends to fully participate in the design and implementation of such a system. A similar strategy has been successfully deployed by the LHC in the form of the Worldwide LHC Computing Grid (WLCG)~\cite{SHIERS2007219}. ECCE has developed and, during the EIC detector proposal period, deployed a tiered ``Butterfly'' model for EIC computing that was inspired by the WLCG model, but updated to better reflect the computing landscape anticipated for the EIC. In this model, the EIC detector supplies the data, but the SDCC at BNL is treated as one of a pool of sites used for long term storage and compute resources. Both BNL and JLab would be considered as \emph{Echelon 1} sites with the ability to add others as appropriate. Raw data would be distributed amongst multiple \emph{Echelon 2} sites for processing with the processed data being returned to Echelon 1. Researchers would directly access the processed data at the Echelon 1 sites.

We have adopted a fixed-latency offline computing model where both the final calibration and reconstruction of raw data occur within 2-3 weeks of acquisition. During this period, raw data will be buffered on disk at all of the Echelon 1 sites, along with permanent archival copies on tapes. Final calibration will be performed semi-automatically including accumulating sufficient data for tracker alignment and energy scale calibration of the calorimeters. Artificial intelligence and machine learning will be integrated throughout this model. After calibration, data processing  will be released to multiple sites including HTC facilities at both Echelon 1 and 2 sites. The EIC will also require large simulation samples to aid in understanding the detector response and physics and background processes being measured.
 We expect that the produced simulation sample will focus on 10\% of the EIC collision cross-section that is directly relevant for the signal and background of the core ECCE physics program. These physics processes will be simulated to $O(10)$~times the statistics in real data to constrain systematic uncertainty from the simulated sample to be much smaller than the data statistical uncertainty.

A summary of the anticipated resource requirements can be seen in table \ref{tab:computing-integrated_luminosity_by_year}.


\begin{table*}[ht]
    \centering
    %\hskip-0.8cm
    \begin{tabular}{c|c|c|c}
        \hline
        % \hline
         \textbf{ECCE Runs}       & year-1                & year-2                  & year-3                \\
        \hline
         Luminosity              & $10^{33}\mathrm{cm}^{-2}\mathrm{s}^{-1}$ & $2\times 10^{33}\mathrm{cm}^{-2}\mathrm{s}^{-1}$ & $10^{34}\mathrm{cm}^{-2}\mathrm{s}^{-1}$ \\
        %  \hline
         Weeks of Running        & 10                    & 20                      & 30                    \\
        %  \hline
         Operational efficiency    & 40\%                  & 50\%                    & 60\%                  \\
        %  \hline
        Disk (temporary)  &  1.2PB & 3.0PB & 18.1PB \\
        %\hline
        Disk (permanent)    & 0.4PB & 2.4PB &	20.6PB \\
        %\hline
        %\textbf{TOTAL}          & 1.6PB &	5.4PB &	38.7PB \\
         Data Rate to Storage    & 6.7Gbps               & 16.7Gbps                & 100Gbps               \\
        %  \hline
         Raw Data Storage (no duplicates) & 4PB          & 20PB                    & 181PB                 \\
        %  \hline
        %  Total Storage (no duplicates) & 4.4PB           & 22PB                   & 200PB                  \\
        %  \hline
        \hline
        Recon process time/core	& 5.4s/ev	& 5.4s/ev	& 5.4s/ev \\
        % \hline
        Streaming-unpacked event size	& 33kB	& 33kB & 33kB \\
        % \hline
        Number of events produced &	121 billion	& 605 billion & 5,443 billion \\
        % \hline
         Recon Storage          & 0.4PB                  & 2PB                    & 18PB                   \\
        %  \hline
        % CPU-core hours (recon-only, 1 pass)	& 24Mcore-hrs	& 121Mcore-hrs &	1089Mcore-hrs \\
        % \hline
        % CPU-core hours (calib-only) &	1.2Mcore-hrs &	6.0Mcore-hrs &	54.4Mcore-hrs \\	
        % \hline
        CPU-core hours (recon+calib)	& 191Mcore-hrs	& 953Mcore-hrs &	8,573Mcore-hrs \\
        % \hline
        2020-cores needed to process in 30 weeks	& 38k &	189k &	1,701k \\
        % \hline
        \hline
   \end{tabular}
    \caption{Estimate of raw data storage and compute needs for first 3 years of ECCE,  assuming ramp up to full luminosity by year 3. 
    % Numbers for the first two years are estimated for the purposes of this exercise and do not come from an external source. n.b. each value represents \emph{only} the needs for data produced in that year and \emph{not} a cumulative total.
    }
    \label{tab:computing-integrated_luminosity_by_year}
\end{table*}
