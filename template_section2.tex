The ECCE detector is a cylindrical detector covering
$\left|\eta\right| \leq 1.1$ and the full azimuth.  It is designed to use
the former BaBar superconducting solenoid to contain
an inner tracking system out to 80~cm in radius followed by an
electromagnetic calorimeter and the first of two longitudinal segments of
a hadronic calorimeter, which is not instrumented in the project baseline.  The second
longitudinal segment of the hadronic calorimeter, which is instrumented to
$|\eta| \leq 1.1$, also serves as
the magnet flux return, surrounding the magnet cryostat.

\subsection{Acceptance}

The large acceptance and high rate of ECCE are key enablers of the ECCE 
physics program~\cite{Adcox:2001jp}.
%%detailed in Chapter~\ref{sci_obj_perf}.  
The total
acceptance of the detector is determined by the requirement of high
statistics jet measurements and the need to fully contain both single
jets and dijets.  To fully contain hadronic showers in the detector
requires both large solid angle coverage and a calorimeter deep enough
to fully absorb the energy of hadrons up to 70~GeV.


\subsection{Second subsection}
